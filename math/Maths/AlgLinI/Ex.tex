\documentclass[a4paper]{article}

\usepackage[utf8]{inputenc}
\usepackage[T1]{fontenc}
\usepackage{textcomp}
\usepackage[french]{babel}
\usepackage{amsmath, amssymb}
\usepackage[a4paper,text={0.8\paperwidth, 0.8\paperheight},centering]{geometry}
\usepackage{bbm} %RCQ
\usepackage{mathtools}
\newcommand{\R}{\mathbbm{R}}
\newcommand{\inc}{\subset}
\newcommand{\defeq}{\vcentcolon=}
\newcommand{\ra}{\Rightarrow}
\newcommand\scal[1]{\left\langle#1\right\rangle}
\newcommand\Norm[1]{\left\lVert#1\right\rVert}
\newcommand\n{\newpage}
% figure support
\usepackage{pdfpages}
\usepackage{transparent}

\pdfsuppresswarningpagegroup=1
\begin{document}
Suppose $\lambda$ is an eigenvalue of $T$ and $v$ a corresponding eigenvector. Then

$$ \begin{aligned} 0 &= (T - 2I)(T - 3I)(T - 4I)v\\ &= (T^3 - 9T^2 + 26T - 24I)v\\ &= T^3 v - 9T^2 v + 26T - 24v\\ &= \lambda^3 v - 9\lambda^2 v + 26\lambda v - 24v\\ &= (\lambda^3 - 9\lambda^2 + 26\lambda - 24)v\\ \end{aligned} $$
\\



If $\operatorname{null} T = \{0\}$ (because it implies surjectivity) or $\operatorname{range} T = \{0\}$ the result is obvious. Assume both contain non-zero vectors.

Since $\operatorname{null} T \neq \{0\}$, we have that $0$ is an eigenvalue of $T$ and that $E(0, T) = \operatorname{null} T$. Let $\lambda_1, \dots, \lambda_m$ denote the other distinct eigenvalues of $T$. Now 5.41 implies that $V = \operatorname{null} T \oplus E(\lambda_1, T) \oplus \dots \oplus E(\lambda_m, T)$. We will prove that $\operatorname{range} T = E(\lambda_1, T) \oplus \dots \oplus E(\lambda_m, T)$.

Suppose $v \in E(\lambda_1, T) \oplus \dots \oplus E(\lambda_m, T)$. Then $v = v_1 + \dots + v_m$ for some $v_1, \dots, v_m$, where each $v_j \in E(\lambda_j, T)$. Moreover, $v = T(\frac{1}{\lambda_1}v_1) + \dots + T(\frac{1}{\lambda_m}v_m)$, which implies that $v \in \operatorname{range}(T)$. Hence

$$E(\lambda_1, T) \oplus \dots \oplus E(\lambda_m, T) \subset \operatorname{range} T.$$

For the inclusion in the other direction, suppose $v \in \operatorname{range} T$. Note that $\operatorname{range} T$ stays the same when we restrict $T$ to $E(\lambda_1, T) \oplus \dots \oplus E(\lambda_m, T)$. Then $v = T(v_1 + \dots + v_m)$ for some $v_1, \dots, v_m$, where each $v_j \in E(\lambda_j, T)$. Therefore $v = \lambda_1 v_1 + \dots + \lambda_m v_m$ and, because $\lambda_j v_j \in E(\lambda_j, T)$ for each $j$, this proves that $v \in E(\lambda_1, T) \oplus \dots \oplus E(\lambda_m, T)$. Thus $\operatorname{range} T \subset E(\lambda_1, T) \oplus \dots \oplus E(\lambda_m, T)$, completing the proof.
\n

$$ \begin{aligned} \operatorname{dim} \operatorname{null} T + \operatorname{dim} \operatorname{range} T &= \operatorname{dim} V\\ &= \operatorname{dim} (\operatorname{null} T + \operatorname{range} T)\\ &= \operatorname{dim} \operatorname{null} T + \operatorname{dim} \operatorname{range} T - \operatorname{dim}(\operatorname{null} T \cap \operatorname{range} T)\\ \end{aligned} $$

Therefore $\operatorname{dim}(\operatorname{null} T \cap \operatorname{range} T) = 0$, implying (c) is true.

Suppose (c) holds. By 1.45, $\operatorname{null} T + \operatorname{range} T$ is a direct sum. Then, by 2.43 and 3.22, we have

$$ \operatorname{dim} (\operatorname{null} T \oplus \operatorname{range} T) = \operatorname{dim} \operatorname{null} T + \operatorname{dim} \operatorname{range} T = \operatorname{dim} V $$

Since $\operatorname{null} T \oplus \operatorname{range} T$ is a subspace of $V$, it follows that $\operatorname{null} T \oplus \operatorname{range} T = V$, implying (a) and completing the proof.
\\


$$ \begin{aligned} STv_j &= S(\lambda_j v_j)\\ &= \alpha_j \lambda_j v_j\\ &= \alpha_j Tv_j\\ &= T(\alpha_j Tv_j)\\ &= TSv_j\\ \end{aligned} $$
\n
For the converse, we will prove the contrapositive, that is, if $\langle u, v \rangle \neq 0$, then $||u|| > ||u + av||$ for some $a \in \mathbb{F}$.

Suppose $\langle u, v \rangle \neq 0$. Note that neither $u$ nor $v$ can equal $0$. We have

$$ \begin{aligned} ||u + av||^2 &= \langle u + av, u + av \rangle\\ &= \langle u, u \rangle + \langle u, av \rangle + \langle av, u \rangle + \langle av, av \rangle\\ &= ||u||^2 + \langle u, av \rangle + \langle av, u \rangle + |a|^2 ||v||^2\\ &< ||u||^2 \end{aligned} $$

where the last line follows provided that $\langle u, av \rangle + \langle av, u \rangle + |a|^2 ||v||^2 < 0$. By 6.14, we can write $u = cv + w$ for some $c \in \mathbb{F}$ and $w \in V$ such that $\langle v, w \rangle = 0$. Note that $c \neq 0$, because $\langle v, v \rangle \neq 0$ and

$$ 0 \neq \langle u, v \rangle = \langle cv + w, v \rangle = \langle cv, v \rangle + \langle w, v \rangle = c \langle v, v \rangle $$

Choose $a = -c$, then

$$ \begin{aligned} \langle u, av \rangle + \langle av, u \rangle + |a|^2 ||v||^2 &= \langle cv + w, av \rangle + \langle av, cv + w \rangle + |a|^2 ||v||^2\\ &= \langle cv, av \rangle + \langle w, av \rangle + \langle av, cv \rangle + \langle av, w \rangle + |a|^2 ||v||^2\\ &= c\bar{a} \langle v, v \rangle + a\bar{c} \langle v, v \rangle + |a|^2 ||v||^2\\ &= (c\bar{a} + a\bar{c} + ||c||^2) ||v||^2\\ &= (-c\bar{c} - c\bar{c} + ||c||^2) ||v||^2\\ &= - |c|^2 ||v||^2\\ &< 0\\ \end{aligned} $$
\n
$$ \begin{aligned} \operatorname{dim} \operatorname{null} T^* &= \operatorname{dim} (\operatorname{range} T)^\perp\\ &= \operatorname{dim} W - \operatorname{dim} \operatorname{range} T\\ &= \operatorname{dim} W + \operatorname{dim} \operatorname{null} T - \operatorname{dim} V \end{aligned} $$

where the first line follows from 7.7 (a), the second from 6.50 and the third from 3.22. We alse have

$$ \begin{aligned} \operatorname{dim} \operatorname{range} T^* &= \operatorname{dim} (\operatorname{null} T)^\perp\\ &= \operatorname{dim} V - \operatorname{dim} \operatorname{null} T\\ &= \operatorname{dim} \operatorname{range} T \end{aligned} $$

where the first line follows from 7.7 (b), the second from 6.50 and the third from 3.22.
\end{document}

